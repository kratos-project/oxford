\documentclass{article}
\usepackage[utf8]{inputenc}
\usepackage{geometry}
\usepackage{float}
\usepackage[final]{graphicx}
\usepackage{grffile}
 \geometry{
 top=30mm,
 left=1.5in,
 right=1.5in
 }
\PassOptionsToPackage{hyphens}{url}\usepackage{hyperref}


\title{Kratos: A solution for data privacy, literacy and student agency in a data-driven educational ecosystem}
\author{Dr.\ Velislava Hillman \quad Varunram Ganesh}
\date{June 2019}

\begin{document}

\maketitle

\textit{Abstract}
Growing digitization has made data ownership an important focus point for institutions and students. Broadly, there are three issues, which require urgent attention for optimization of data privacy, literacy, and utilization. First, schools globally are equivocal about data generated by and about students as a result of the digitization of instruction, learning, and assessment. They lack necessary frameworks for data literacy, data interoperability, and optimization while maintaining privacy and control. Second, the scale, source, and nature of school data makes its interoperability impractical, resulting in an inability to assess the true impact of educational technologies on instruction and learning. Third, while data helps teachers improve pedagogical practices, an increasingly data-driven decision-making process suggests that student dimensions of learning and equitable participation in curriculum design becomes secondary. Finding a balance between data-driven decision making and student voice is critical for an efficient educational ecosystem. In this paper, we introduce Kratos: an immutable decentralised data management system that provides data privacy and applied data literacy while empowering students with a user interface for data governance and active participation in the educational ecosystem. Using the advantages of blockchain technologies, Kratos enables easy authentication and access to data. The objective of Kratos is thus to equip schools and students with the ability to access, manage and control their data and to understand how, why, and by whom data is accessed without compromising student agency and privacy. This paper describes proof of concept for Kratos and its benefits to the educational ecosystem.
\bigbreak

\section{Introduction}

To improve work, school practitioners at local, district, state, and federal levels need data interoperability and educational information from data mining [Gates Foundation, 2015; US Dept of Education, 2012; ACT Policy Report, 2015]. This need originally emanates from the United States federal law which mandates schools at district, state, and federal level to collect student data to serve as 'accountability' metric for assessing school effectiveness [Elementary and Secondary Education Act, 2001; 2010].
\bigbreak
Educational data is important not only for evaluating school quality. It can also contribute to research in learning and advance theory [Baker, 2016]. It can help guide intervention and lead to higher school attainment [Arnold and Pistill, 2012], improved pedagogy [Baker, 2014], and better parental involvement in school matters [Hawn, 2015]. A common goal of educational data is to equip educators with key information that they can act upon and use to the benefit of learners [Baker, 2014]. Educational data can enable an applied data literacy. Furthermore, access to school records provides an opportunity for student agency by enabling learning and participation.
\bigbreak
Increasingly digitized learning environments enable constant data collection and algorithm-based assessment which change core school functions of teaching, assessment, and accreditation [Zeide, 2017]. Digital systems for teaching and assessment drive data-based decision-making and enable micromanagement of students that can further pose restrictions on teacher autonomy and student (and parent) participation and willingness to challenge education decision-making. Schools deploy management systems for sorting and storing data and while some data is stored by schools at the district level, other data is managed by third-party vendors [Zhang, Li and Hao, 2015] who may consider this data proprietary and thus inaccessible. This puts barriers to effective data sharing and comprehensive understanding of school data.
\bigbreak
In this work we propose the development of a structure for student data management, privacy, accountability, and auditability. We build on top of existing data standards and construct a common data schema for disparate data across different systems. Our system organises these references onto an integral structure of student data and complements the existing educational data standards and interoperability in three distinctive ways. Through network permissioning and proofs of ownership on a distributed ledger, we enable data auditability and accountability. We design data analytic models to integrate with existing school systems and data standards. Lastly, we build a simple user interface that gives students, parents and schools access to disparate data, enabling control over what portion of the data can be shared. Additionally, we propose an application with social functionality for student feedback.
\bigbreak
We present Kratos not as an independent solution to lack of data interoperability, accountability, and privacy but rather as a system that complements existing efforts and solutions developed by the various stakeholders in the educational ecosystem (Data Quality Standards, 2018; Project Unicorn; Bill and Melinda Gates Foundation, 2015).

\section{Existing problems}
To contextualise the complexity related to school data - the opportunities and challenges schools face as a result of the growing digitisation of operation and academic processes - Kratos has partnered with  Cambridge Public School (CPS) district that administer public elementary and high schools in Cambridge, Massachusetts, the Access for Learning (SIF), and the Student Data Privacy Consortium. With the help of CPS' Information and Communication Technology Services Chief Information Officer and Database Administrator, we explored data interoperability issues with them along with feedback on potential mitigation schemes:

\begin{itemize}
  \item \textbf{Data access} - CPS is in agreement with over 100 vendors providing education technologies. A majority of these provide no direct and comprehensive access to data generated about and by students who use of their products. Where available, data access is  provided in the form of reports - a summary of information which the district database administrator can request and download. On other occasions vendors supply data directly to teachers as well as the students in the form of digital dashboards. Where teachers obtain data it requires further work to convert it into meaningful information upon which the teacher can adjust and plan instruction (Data Quality Campaign, 2018).
  
  \item \textbf{Lack of data standard compliance} - School data frameworks vary across districts and states. Additionally, vendors providing education technologies use different data formats, schema and elements to organise and store student data as no single standard can be enforced upon them to comply with. Thus, disparate data scattered across different systems with varying degrees of accessibility and usability poses challenges to educators and other stakeholders in the education sector in assessing the impact of education technologies on the learning process. Moreover, the lack of data standard compliance further limits the potential for imposing privacy law as is the case with the General Data Protection Regulation (GDPR, 2016), which provides sound privacy practices that are, on the one hand, enabled by data interoperability, and on the other, imposing vendors to maintain data transparency.
  
  \item \textbf{Lack of transparency} - The scale, complexity, and number of providers pose challenges for schools to have a comprehensive list of the data that is generated about students. Lack of data interoperability standards which would simply be the process of compiling, organising and documenting what information is generated about students makes data auditability difficult to achieve, less so to be useful to the instruction and learning processes. 
  
  \item \textbf{Security} - Due to lack of transparency with regards to how various education technology vendors organise and structure their data, it becomes difficult to understand the security they ensure for the data their products and services generate. A recent case  presents the fears that exist with regards to school data (Cameron, 2017) - Schoolzilla, a K-12 education data service provider, detected a data breach which exposed the personal data of 1.3 million K-12 students. Some argue (Hand, 2018) that currently available cybersecurity protocols are inefficient to secure student data while at the same time allowing data analytics. Having inadequate access to available data further limits potentials to fully benefit from it.
  
\end{itemize}

We thus infer that schools are finding it increasingly difficult to obtain data about their own students in a comprehensive and timely manner and are thus becoming isolated due to a lack of knowledge on how vendors are processing the data that is generated as a result of their products and services used in class.

\section{The need for a solution}

Interoperability challenges between vendors and schools at local, district, state, or federal level pose barriers to cohesive data management and sharing while lacking basic technological privacy infrastructure and accountability (Zeide, 2014; Common Sense Media Research, 2018). While digitizing and collecting student data is not new (Fitzgerald, 2014), technological developments like cloud computing and the Internet of Things (IoT) amplify concerns about data transfer, storage, use, and analysis (Sultan, 2010).
\bigbreak
Concerns arise about over-prevalence of fine-grained data collected about students (Zeide, 2017) and the risk of creating a 'permanent record' that can impede upon learners' futures (Cody, 2013). The use of online platforms and applications in schools, often provided by for-profit vendors many of whom have unclear policies about data privacy (Common Sense Education, 2018), generates a continuous stream of data about students' behavior and performance at an unprecedented scale (Zeide, 2017). Big data complicates traditional understanding of what constitutes sensitive information and what information serves an educational purpose. Such data however, continues to drive decision-making of practitioners and vendors, marginalizing student dimensions of learning and equitable participation in curriculum design.
\bigbreak
The present challenges of fragmented data access, the lack of student agency and auditability and the lack of a concrete framework for data accessibility demand a socio-technological solution which can ensure that educational data can be used to help improve school processes while not diminishing student privacy, agency and future opportunities.

\section{Kratos}
Kratos aims to define a set of guidelines and principles that schools should follow and provides a platform for both schools and students to see what kind of data is being collected by vendors with the help of an immutable log for auditing student data changes, access, and use. The goal for students would be to see how their data is being used and above the age of 18, gain access to this data. We define this as Data Ownership and Visibility - how students can own the data after the age of 18 and how students can see what their data is being used for. The goal for the school would be to make disparate data from different vendors compatible with each other and we define this problem as Data Interoperability.

\begin{figure}[H]
    \centering
    \includegraphics[width=15cm]{{"images/Kratos.jpg}}
    \caption{Kratos Architecture}
    \label{fig:my_label}
\end{figure}

\subsection{Data Interoperability}
A traditional approach to interoperability would be to force a given set of standards on third party vendors, so that they reference them. However, work done by standardization bodies like SIF and CEDS has shown that this is at best partially effective since there is no financial or social incentive for vendors to migrate to the proposed set of standards. Kratos attempts to solve this issue by proposing a solution where different fields described by different vendors can be mapped to a single underlying scheme defined by Kratos and third party vendors would not have to change their established data organisation and structure.

\begin{figure}[H]
    \centering
    \includegraphics[width=15cm]{{"images/data-architecture.jpg}}
    \caption{Data Architecture}
    \label{fig:my_label}
\end{figure}

\bigbreak
As an example, let us assume the underlying fields defined by Kratos are Grade\_A, Grade\_B and Grade\_C. If a specific vendor has a different name for the same field (example GradeA, GradeB and GradeC), Kratos would ingest the data from the vendor defined fields and convert it to the ones defined by Kratos. This should be done across different formats since the reports database administrators receive right now are formatted as JSON, CSV or are in the form of Excel sheets.

\bigbreak
Kratos proposes that this be done in the form of a templated script that can be written for every vendor. Building a common script for all the vendors poses a challenge since the number of vendors and the different fields and standards they follow are ever changing. This templated script would be triggered automatically by Kratos each time it receives an incoming report and the report itself will be parsed to understand which vendor it originates from.

\bigbreak
The fields that Kratos defines will be adapted from the existing SIF standard followed by schools in Cambridge and would also have routine input from various experts on the subject. The number of fields however would change since SIF has over 700 different fields and it is not possible to accurately map all these fields. Kratos defines the idea of a "bucket" - a collection of data fields collated together as a single entity to make it easier for students and administrators to monitor them. Common examples of buckets would be PII (Personally Identifiable Information), Grade Reports (containing grade reports for each subject) and Attendance Records (recording the class wise attendance of the student). To formalise the type of data buckets it is imperative to develop a data taxonomy that comprises the existing school data standards and data elements according to which different education technology providers collect and organise their data. 

\bigbreak
The models and scripts developed as part of enabling interoperability will be open source and subject to continuous updates. We would also be having a complete code audit before releasing the system in a production environment to ensure the model performs and behaves the way it was designed and intended to.

\bigbreak
The buckets defined by Kratos can be used in multiple ways: students can have a better understanding of how their data is being used, parents can know what kind of data is being collected about their children, and administrators can ensure that sensitive information is not being shared with third-party vendors. After legal age of consent, this control would be given to the hands of students and they can decide if they wish to continue sharing data with these parties.

\subsection{Data control}
In conventional systems, data control or 'ownership' can be proved with the help of an access token but this provides no guarantee on when the owner came into possession of the data. In order to attest 'ownership' at a specific point in time, we need time-stamping services like those described in \hyperref[sec:1]{[1]}, \hyperref[sec:2]{[2]} and \hyperref[sec:3]{[3]}. A time-stamping service requires something to be committed along with the time-stamp and Kratos envisions this to be a cryptographic hash which also acts as an access token which vendors can use to access any student-generated data.
\bigbreak
The aim of using a cryptographic hash like SHA-3 \hyperref[sec:6]{[6]} is to ensure a uniquely random reference to the data to avoid out-of-channel data leaks. Kratos suggests using an element of randomness like a salt to generate the hash in order to have the ability to revoke tokens by regenerating randomness. Kratos enforces that all schools encrypt their data before creating access tokens to mitigate the risk of loss/theft of data. Past studies like \hyperref[sec:4]{[4]} and \hyperref[sec:5]{[5]} show that firms are willing to circumnavigate laws to collect data and encrypting data by design ensures that no third party can have access without being granted so explicitly.
\bigbreak
Kratos suggests that encryption be done at the student level but this can also be at the school level depending on existing frameworks and rules surrounding the school. If the school chooses to encrypt student data on behalf of the student, Kratos enforces that the school use a unique key for each individual to minimize the risk of key theft. In addition to this, Kratos suggests that schools store their encrypted data in a distributed file storage system like IPFS to ensure data redundancy and availability in case of a setback. Storing data on IPFS also makes it easier to create timestamps since it is sufficient to reference the IPFS pointer instead of potentially hashing the whole data.
\bigbreak
In the event a user wants to revoke access to a particular vendor, he could do so by changing the encryption key or by changing the randomness used to generate the access token. We suggest users do not regenerate encryption keys but Kratos will provide users an option to choose between the two.
\bigbreak
Since schools have their own sets of policies, Kratos does not strictly enforce a set of practices for users to follow. This ensures that adoption of Kratos is not constrained by a certain set of rules. At the same time, Kratos defines a set of minimum requirements to be on board to ensure good practices on data protection are followed.
\bigbreak
Kratos also does not enforce how commitments need to be generated and stored, leaving it as an option for schools to provide their feedback on. Solutions provided by Kratos would include a centralized time-stamping server, a permissioned blockchain with the different schools as the nodes and simple time-stamping commitments to an existing blockchain. All three have their benefits and constraints and Kratos would enable schools to customize this to their requirements.
\bigbreak

\subsection{Student agency and participation}
The UNESCO framework for educational planning states that "the concern of planners is twofold: to reach a better understanding of the validity of education in its own empirically observed specific dimensions and to help in defining appropriate strategies for change" (Haddad, 1995, pp. 5-6).
\bigbreak
While summative and cumulative assessments provide "empirically observed specific dimensions" about student academic performance, student agency and active participation is equally required in order for policy and education to design "strategies for change". Growing use of education technologies enables data-driven decision-making [Gibson et al., 2015]. Technology-mediated instruction and assessment tools with learning analytics functionality track and diagnose student progress. Most educational technologies can interpret and equip educators with information via digital dashboards and 'skill meters'[Baker, 2016; New, 2016]. Fine-grained and continuously accumulated data about student behavior and performance surpasses traditional notions of assessment [Zeide, 2017] posing limitations over student dimensions of learning and equitable participation in the curriculum design.
\bigbreak
Our prototype provides a graphical interface for student involvement in and accessibility to school data. While students and equally their parents can become acquainted with any changes and meanings of school data, students can also participate with personal feedback to the learning process. Student participation with personal perspectives and reflection are integral to the learning process (Ackermann, 1996). Some reflection is invariably carried out through questionnaire surveys examining school climate (Holahan and Batey, 2019). Both school climate surveys and the proposed student feedback application provide flexible selection of questionnaires and measurements with a common goal to improve school climate. However, the proposed application enables not only feedback from older students (Ibid.) but from all students, provided that the feedback addresses the goal to encourage perspective-taking and reflection that are deemed necessary to the learning process (Kegan, 1982). The application further enables student agency and control over the frequency, depth, and nature of the feedback provided that it directly reflects the learning experience.

\begin{figure}[H]
    \centering
      \begin{minipage}[b]{0.4\textwidth}
    \includegraphics[width=5cm]{{"images/home.png}}
    \caption{Student UI: Student view of data information}
    \label{fig:my_label}
    \end{minipage}
    \begin{minipage}[b]{0.4\textwidth}
    \includegraphics[width=5cm]{{"images/summary.png}}
    \caption{Student UI: Student view of detailed data fields}
    \label{fig:my_label}
    \end{minipage}
\end{figure}

\subsection{Data literacy}
 Data literacy is still a debatable construct (Bowler and Acker, 2017). Data means quantified information, which is "situated, taking its meaning from its context and the perspective of its beholder" (Borgman, 2015, p. 18). Making sense of data is relative. What data may mean to someone can mean nothing to others (Bowler and Acker, 2017). In increasingly data-driven systems it becomes imperative not only to understand data but to interact with it and participate in the decision-making processes that it increasingly begins to impact. While data provides insight about how systems operate, the inferences drawn from it is without context. Fine-grained data collected over long periods of time can lead to negative impact on individual well-being and pose limitations over future opportunities (Altman et al., 2018; Friedman and Nissenbaum, 1996). 
 \bigbreak
 The growing digitization of schools create learning environments that enable constant data generation, the access to, and use of which becomes hard to control, audit, account for, and understand its impact. The risk from digitized school environments that constantly generate data into a permanent record (Sirota, 2013) that at any one point in time may limit individual opportunities leads us to prioritise on developing a techno-social solution that enables applied data literacy for students at an ever younger age.
 \bigbreak
 While some studies examine aspects of data literacy in young people's lives (Bowler and Acker, 2017; Selwyn and Pangrazio, 2018; Kumar et al., 2017), the focus remains on Internet and social media data. Studies related to young people's perceptions, knowledge, and understanding of school data literacy are meagre. 
 \bigbreak
 Existing literature demonstrates that young people have varying interprations and a general understanding of data. However, many still find difficulties to connect with data at a concrete level, with the notion of having a data dossier (Bowler and Acker, 2017, p. 27). This highlights the pressing need to develop strategies for school-related data literacy, introduce data management skills, and encourage student participation in an increasingly data-driven decision-making school environment.
 \bigbreak
 Kratos thus proposes a comprehensive, yet simple access, knowledge, and interaction with school data. At conceptualisation stage we choose three core data elements to work with. These are assessment, including test scores, reports, and credentialing; conduct, including observation reports, behaviour, and attendance; and personally identifiable information, including personal identifiers, demographics, parent information, medical reports, and socio-economic information. We take these three data elements crucial to student privacy and control. In our conceptual work (Figures 3-5) we recognise that while schools collect data for each of these three elements, digital applications used in class also collect similar such data. For example, as schools contain data about conduct, applications such as Class Dojo (Manolev et al., 2018) gather similar data. Teachers may further report behaviour using Swivl. Currently, students do not have a comprehensive access to or knowledge of all data that exists about their conduct. Moreover, there is no awareness and knowledge of the potential impact of all data that exists about their conduct. We therefore create visibility to the disparate data that exists about student conduct, provide visibility and control over who accesses such data and for what purpose, and enable learning about data (figure 6), its meaning and purpose. 

\begin{figure}[H]
    \centering
      \begin{minipage}[b]{0.4\textwidth}
    \includegraphics[height=9.5cm]{{"images/data-detail-1.png}}
    \caption{User UI: Student view of data}
    \label{fig:my_label}
    \end{minipage}
      \begin{minipage}[b]{0.4\textwidth}
    \includegraphics[height=9.5cm]{{"images/data-detail-2.png}}
    \caption{User UI: Student view of data and definition}
    \label{fig:my_label}
    \end{minipage}
\end{figure}

\begin{figure}[H]
    \centering
      \begin{minipage}[b]{0.4\textwidth}
    \includegraphics[height=9.5cm]{{"images/detail.png}}
    \caption{Permit / Revoke Access}
    \label{Permit / Revoke Access}
    \end{minipage}
      \begin{minipage}[b]{0.4\textwidth}
    \includegraphics[height=9.5cm]{{"images/feedback-1.png}}
    \caption{Feedback}
    \label{fig:my_label}
    \end{minipage}
\end{figure}

\section{Discussion and future work}
In this paper we propose Kratos, a decentralised data management system which enables data interoperability, applied data literacy, and student participation in the curriculum design. This proposal envisions an ideal techno-social solution that empowers students by providing more control over their school data through data transparency, accountability and auditability. At the same time, continuous legal (GDPR, 2016; Colorado Student Data Transparency and Security Act, 2016; Kelly, 2019) and policy (Student Data Privacy Consortium, 2018) efforts are being made that will further help establish Kratos as a tool integral to the school ecosystem.
\bigbreak
Unlike other sectors such as retail (Lu and Xu, 2017), finance (Ito et al., 2017), and medicine (Acbo et al., 2019), education is still not fully leveraging the power of global digital transformation. In medicine, for instance, data interoperability is important in order for various stakeholders to access comprehensive patient records and various cases across Europe (Demertzis et al., 2018) and the United States (Azaria et al., 2016) decentralised architectures have been developed to enable data sharing, interoperability, and access while prioritising patient agency, privacy, and security. Similarly, a more comprehensive data management for interoperability is highly needed in the education sector without compromising student privacy and control over their data.
\bigbreak
On the other hand, while data analytics provides insight about school processes, not everything that is measured is meaningful. Students’ perspectives and reflection should equally count next to data-driven decision-making. Moreover, the sophistication of data analytics and long-term data collection further decrease an individual's privacy. The risk of being publicly exposed can lead to being embarrassed or discriminated (Altman et al. 2018). Constantly being aware of such risks can diminish one’s sense of freedom for self-expression and in the case with children and young people - the freedom to try out new things and make mistakes. Therefore, a decentralised data management system that prioritises student privacy and control over data, becomes crucial to the development of a safe and free learning environment.
\bigbreak
Future work includes three distinctive steps. First, we plan to develop a comprehensive taxonomy of school data. This will serve to develop a comprehensive data vocabulary for applied data literacy and identify data elements. The second step involves prototyping and formalising platform sections and functionality. And the third involves carrying out user studies to get student feedback and the fine tuning of the platform's user interface.

\section{Acknowledgements}
We thank Cambridge Public School district, Student Data Privacy Consortium, Access for Learning (SIF) for their support. We would also like to thank Rayner Ng Jing Kai and Geoffrey Martin from Yale-NUS College for their help in sketching mock-up designs and giving feedback on data interoperability.

\section{References}
\begin{enumerate}
    \label{sec:1}
    \item Todd, P. OpenTimestamps: Scalable, Trustless, Distributed Timestamping with Bitcoin (2016). URL \url{https://petertodd.org/2016/opentimestamps-announcement}
    \label{sec:2}
    \item Szalachowski, P. (2018). Towards more reliable Bitcoin timestamps. arXiv preprint arXiv:1803.09028.
    \label{sec:3}
    \item Massias, H., Avila, X. S., \& Quisquater, J. J. (1999). Design of a secure timestamping service with minimal trust requirement. In the 20th Symposium on Information Theory in the Benelux.
    \label{sec:4}
    \item Zetter, K. Google Collected Data on Schoolchildren without permission (2016). URL \url{https://www.wired.com/2015/12/google-collected-data-on-schoolchildren-without-permission/}
    \label{sec:5}
    \item Dissent, Back-To-School Revolt in Springfield? Employees balk over using Google Drive as evidence of massive privacy breach mounts (2018).  URL \url{https://www.pogowasright.org/back-to-school-revolt-in-springfield-employees-balk-over-using-google-drive-as-evidence-of-massive-privacy-breach-mounts/}
    \label{sec:6}
    \item Bertoni, G., Daemen, J., Peeters, M., Van Assche, G., \& Van Keer, R. (2012). Keccak implementation overview. URL  \url{http://keccak.neokeon.org/Keccak-implementation-3.2.pdf}
    \label{sec:7}
    \item CPSD. District Agreements Listing (2019). URL  \url{https://sdpc.a4l.org/district_listing.php?districtID=457}
    \label{sec:8}
    \item Colins, A., and Halverson, R. (2018).\textit{ Re-thinking education in the age of technology: The digital revolution and schooling in America}. Teachers College Press: New York.
    \label{sec:9}
    Zeide, E. (2016). 19 Times Data Analysis Empowered Students and Schools: Which Students Succeed and Why?.
    \label{sec:10}
    \item Cody, A. (2013). Will the data warehouse become every student and teacher's 'permanent record'? \textit{Education Week}, May, 20.
    \label{sec:11}
    \item Gibson, D. C., Webb, M., and Ifenthaler, D. (2015). Challenges of big data in educational assessment.\textit{Proceedings of the IADIS International Conference of Exploratory Learning in Digital Age}, 92-100.
    \label{sec:12}
    \item Sultan, N. (2010). Cloud computing for education: a new dawn? 30 International Journal of Information Management  109.
    \label{sec:13}
    \item Fitzgerald, B. (2014). Data collection isn't new. And it predates common core.Funny Monkey.
    \label{sec:14}
    \item Keegan, R. (1982). The evolving self: Problem and process in human development. Cambridge: Harvard UP.
    \label{sec:15}
    \item Ackermann, E. (1996). \textit{Constructionism in practice: designing, thinking, and learning in a digital world}. Routledge.
    \label{sec:16}
    \item Gibson, D. C., Webb, M. E., & Ifenthaler, D. (2019). Measurement Challenges of Interactive Educational Assessment. In \textit{Learning Technologies for Transforming Large-Scale Teaching, Learning, and Assessment} (pp. 19-33). Springer, Cham.
    \label{sec:17}
    \item Agbo, C. C., Mahmoud, Q. H., & Eklund, J. M. (2019, June). Blockchain Technology in Healthcare: A Systematic Review. In \textit{Healthcare} (Vol. 7, No. 2, p. 56). Multidisciplinary Digital Publishing Institute.
    \label{sec:18}
    \item Lu, Q., & Xu, X. (2017). Adaptable blockchain-based systems: a case study for product traceability. \textit{IEEE Software,} 34(6), 21-27.
    \label{sec:19}
    \item Ito, J., Narula, N., & Ali, R. (2017). The blockchain will do to the financial system what the internet did to media. \textit{Harvard Business Review,} 9(March).
    \label{sec:20}
    \item Bienkowski, M., Feng, M., & Means, B. (2012). Enhancing teaching and learning through educational data mining and learning analytics: An issue brief. \textit{US Department of Education, Office of Educational Technology,} 1, 1-57. 
    \label{sec:21}
    \item Regulation, G. D. P. (2016). Regulation (EU) 2016/679 of the European Parliament and of the Council of 27 April 2016 on the protection of natural persons with regard to the processing of personal data and on the free movement of such data, and repealing Directive 95/46. \textit{Official Journal of the European Union (OJ)}, 59(1-88), 294.
    \label{sec:22}
    \item Kelly, M. New privacy bill would give parents an ‘Eraser Button’ and ban ads targeting children (2019). URL \url{https://www.theverge.com/2019/3/12/18261181/eraser-button-bill-children-privacy-coppa-hawley-markey}
    \label{sec:23}
    \item Demertzis, I., Papadopoulos, S., Papapetrou, O., Deligiannakis, A., Garofalakis, M., and Papamanthou, C. (2018). Practical Private Range Search in Depth. \textit{ACM Transactions on Database Systems, 43(1),} 2.
    \label{sec:24}
    \item Altman, M., Wood, A. B., O'Brien, D., & Gasser, U. (2018). Practical approaches to big data privacy over time.
    \label{sec:25}
    \item Bowler, L., Acker, A., Jeng, W., and Chi, Y. (2017). It lives all around us: Aspects of data literacy in teen’s lives. \textit{Proceedings of the Association for Information Science and Technology, 54(1)}, 27–35. DOI:10.1002/pra2.2017.14505401004 
    \label{sec:26}
    \item Friedman, B., and Nissenbaum, H. (1996). Bias in computer systems. \textit{ACM Transactions on Information Systems (TOIS), 14(3)}, 330-347.
    \label{sec:27}
    \item Borgman, C. L. (2015). \textit{Big data, little data, no data: Scholarship in the networked world.} MIT Press.
    \label{sec:28}
    \item Kumar, P., Naik, S., Devkar, U., Chetty, M., Tamara, C., and Vitak, J. (2017). 'No telling passcodes out because they’re private’: Understanding children’s mental models of privacy and security online. \textit{Proceedings of the ACM on Human-Computer Interaction, 1(CSCW)}, 64.
    \label{sec:29}
    \item Azaria, A., Ekblaw, A., Vieira, T., & Lippman, A. (2016, August). Medrec: Using blockchain for medical data access and permission management. \textit{In 2016 2nd International Conference on Open and Big Data (OBD)} (pp. 25-30). IEEE.
    \label{sec:30}
    \item Zeide, E. (2014). The proverbial permanent record.

\end{enumerate}

\end{document}
    
